\chapter{Projektbeschreibung}
Die DSA Daten- und Systemtechnik GmbH entwickelt Soft- und Hardwarelösungen zur Diagnose
elektronischer Fahrzeugkomponenten in der Automobilindustrie und bietet Logistik- und
Qualitätssicherungssystemen für die Flughafenlogistik und Baustoffindustrie an. Eines der
Softwareprodukte ist das Remote Production System Prodis.PlantHUB, ein Umlaufprüfsystem zur
zentralen Steuerung und Überwachung von Produktionsumgebungen. Unter anderem bietet es Monitoring
und Konfiguration durch eine Webanwendung. PlantHUB kommuniziert mit dem Fahrzeug über ein
Vehicle-Communication-Interface (VCI), ein Gerät, welches zu Beginn mit dem Fahrzeug verbunden wird
und über das die Position des Fahrzeugs in der Produktionslinie verfolgt wird. Gelangt das Fahrzeug
in einen Testbereich können automatisch Diagnosetests ausgeführt und dessen Daten ausgelesen und
ausgewertet werden. Werksmitarbeiter können mit mobilen Geräten das VCI scannen, und so Daten über
das Fahrzeug und den aktuellen Test erfahren. \\
Das Backend verfolgt die Microservice-Architektur und ist eventbasiert. Es gibt drei Arten von
Microservices: Services enthalten die digitalen Repräsentationen (Digital Twins) von Fahrzeugen,
VCIs und weiteren Geräten. Sie erzeugen außerdem Events. Agents enthalten die Geschäftslogik des
Systems, reagieren auf die Events und ändern den Status der digital Twins basierend auf den Events.
Adapter sorgen für die Kommunikation mit externen Kundensystemen. Das System wird beim Kunden auf
einem Server installiert. Dieser wählt eine beliebige Konfiguration von Features, die durch Microservices
implementiert sind.