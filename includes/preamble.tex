% !TeX root = PB1.tex

% % % % % % % % % % Packages % % % % % % % % % %
\usepackage{appendix}
\usepackage{xcolor}
\usepackage{graphicx}
\usepackage{lmodern}
\usepackage{framed}
\usepackage[section]{algorithm}
\usepackage{verbatim}
\usepackage[english,ngerman]{babel} %% Deutsche Bezeichnungen im Dokument
%\usepackage[ngerman,english]{babel} %% Englische Bezeichnungen im Dokument
\usepackage{hyperref}
\usepackage{amsmath}
\usepackage{amsfonts}
\usepackage{amssymb}
\usepackage{a4wide}
\usepackage[headsepline, footsepline]{scrlayer-scrpage}
\usepackage{microtype}
%\usepackage{fontspec} %% only if  using xelatex
\usepackage{float}
\usepackage[utf8]{inputenc}
\usepackage{stringstrings}
\usepackage{listings}
\usepackage[notmath]{sansmathfonts}
\usepackage{pdfpages}
\usepackage{tikz}
\usetikzlibrary{shapes,arrows.meta,positioning}
\usepackage{lipsum}
\usepackage[style=alphabetic, alldates=long, giveninits=true, maxbibnames=99, sortcites=true]{biblatex}
\usepackage{booktabs}
\usepackage{setspace}

% % % % % % % % % %  DSA Colors % % % % % % % % % %  
\definecolor{dsablue}{HTML}{2E75BF}


% % % % % % % % % % oft verwendete Farben % % % % % % % % % %
\definecolor{black}{rgb}{0,0,0}
\definecolor{darkgrey}{rgb}{.4,.4,.4}
\definecolor{grey}{rgb}{.7,.7,.7}
\definecolor{white}{rgb}{1,1,1}

\definecolor{blue}{rgb}{0,0,1}
\definecolor{red}{rgb}{1,0,0}
\definecolor{green}{rgb}{0,1,0}

\definecolor{midblue}{rgb}{0,0,.7}
\definecolor{midred}{rgb}{.7,0,0}
\definecolor{midgreen}{rgb}{0,.7,0}

\definecolor{darkblue}{rgb}{0,0,.4}
\definecolor{darkred}{rgb}{.4,0,0}
\definecolor{darkgreen}{rgb}{0,.4,0}

\definecolor{orange}{rgb}{1,.4,0}
\definecolor{cyan}{rgb}{0,1,.8}
\definecolor{purple}{rgb}{.6,0,.6}
\definecolor{lavender}{rgb}{.6,.6,1}
\definecolor{yellow}{rgb}{1,.8,.2}



% % % % % % % % % %  RWTH Colors % % % % % % % % % %  
\definecolor{rwthgrey75}{RGB}{100,101,103}
\definecolor{rwthgrey50}{RGB}{156,158,159}
\definecolor{rwthgrey25}{RGB}{207,209,210}
\definecolor{rwthgrey10}{RGB}{236,237,237}

\definecolor{rwthblue100}{HTML}{00549F}
\definecolor{rwthblue75}{HTML}{407FB7}
\definecolor{rwthblue50}{HTML}{8EBAE5}
\definecolor{rwthblue25}{HTML}{C7DDF2}
\definecolor{rwthblue10}{HTML}{E8F1FA}

% % % % % % % % % %  FH, IT Center and MATSE Colors % % % % % % % % % % 
\definecolor{fhgreen}{HTML}{00B1AC}
\definecolor{itcorange}{HTML}{F6A800}
\definecolor{matseblue}{HTML}{00549F}



% % % % % % % % % %  Kopf- und Fu{\ss}zeile %
\automark[chapter]{chapter}
\clearmainofpairofpagestyles

%\rohead{\normalfont\vfill\headmark}
%\lehead{\includegraphics[height=8mm]{media/images/logo_dsa.png}}
%\ofoot*{\pagemark}

\ohead{\includegraphics[height=8mm]{images/logo_dsa.png}}
\ihead{\normalfont\vfill\headmark}
\cfoot*{\pagemark}
\ofoot*{}

\ModifyLayer[addvoffset=-1ex]{scrheadings.foot.above.line}
\ModifyLayer[addvoffset=-1ex]{plain.scrheadings.foot.above.line}
\ModifyLayer[addvoffset=1ex]{scrheadings.head.below.line}


% % % % % % % % % %  Farboptionen % % % % % % % % % % 
%% Screen ist bunt, Print ist schwarz. Passendes einkommentieren.

% Screen
%\hypersetup{
%  pdftitle={Erster Praxisbericht},
%  pdfauthor={Marco JAN{\SS}EN, 3275330},
%  pdfsubject={Aufbau eines lokalen Caches zu Teilen einer Datenbank},
%  colorlinks=true,
%  linkcolor=midblue,
%  citecolor=midgreen,
%  urlcolor=midred
%}

% Print
%\hypersetup{
%  pdftitle={Erster Praxisbericht},
%  pdfauthor={Marco JAN{\SS}EN, 3275330},
%  pdfsubject={Aufbau eines lokalen Caches zu Teilen einer Datenbank},
%  colorlinks=true,
%  linkcolor=black,
%  citecolor=black,
%  urlcolor=black
%}

% % % % % % % % % %  Umgebung f{\"u}r Beispiele % % % % % % % % % % 
\newlength{\leftbarwidth}
\setlength{\leftbarwidth}{1pt}
\newlength{\leftbarsep}
\setlength{\leftbarsep}{10pt}
\newlength{\leftbarheight}
\setlength{\leftbarheight}{10pt}

\newcommand*{\leftbarcolorcmd}[1]{\color{#1}}% as a command to be more flexible


%%% Useful for marking examples, which are inserted into the text. Usage see below.
\newenvironment{example}[1][black]{%
    \def\FrameCommand{{\leftbarcolorcmd{#1}{\hspace{3pt}  \vrule width \leftbarwidth \relax\hspace {\leftbarsep}}}}%
    \MakeFramed {\vspace{2pt} \advance \hsize -\width \FrameRestore }%
\vspace{-0.5em}}{ %
    \vspace{2pt} \endMakeFramed
}

\renewenvironment{leftbar}[1][black]{%
    \def\FrameCommand{{\leftbarcolorcmd{#1}{\hspace{3pt}  \vrule width \leftbarwidth \relax\hspace {\leftbarsep}}}}%
    \MakeFramed {\vspace{2pt} \advance \hsize -\width \FrameRestore }%
\vspace{-1.5em}}{ %
    \vspace{2pt} \endMakeFramed
}

% % % % % % % % % %  Listings % % % % % % % % % % 
\lstset{
    xleftmargin=24pt,
    xrightmargin=24pt,
    numbersep=8pt,
    basicstyle=\ttfamily\footnotesize,
    columns=fixed,
    showstringspaces=false,
    frame=tb,
    numbers=left,
    language=Java
}
\renewcommand\lstlistlistingname{Listingverzeichnis}
\renewcommand\lstlistingname{Listing}

% % % % % % % % % %  Generelle Einstellungen im Dokument % % % % % % % % % % 
\setlength{\parindent}{0pt}
\renewcommand{\familydefault}{\sfdefault}

% \renewlistof\listoffigures{lof}{Abbildungs-, Tabellen- und Listingverzeichnis}

% % % % % % % % % %  Eigene Kommandos % % % % % % % % % % 
\newcommand\todo[1]{\textcolor{red}{\itshape TODO: #1}}
\newcommand\forceblankpage{\newpage\thispagestyle{empty}\phantom{ }\newpage}



% % % % % % % % % %  BibLaTeX % % % % % % % % % % 
\addbibresource{Praxisbericht2.bib}
\DefineBibliographyExtras{german}{
    \DeclareNameAlias{author}{family-given}
    \DeclareDelimFormat{finalnamedelim}{;\space}
    \DeclareDelimFormat{multinamedelim}{;\space}
    \renewcommand*{\mkbibnamefamily}[1]{\textsc{#1}}
    \DeclareDelimFormat[bib]{nametitledelim}{\addcolon\space}
    \DeclareFieldFormat{urldate}{\addcomma\space abgerufen am\space#1}
}


% % % % % % % % % %  Einstellungen KOMA-Script % % % % % % % % % % 
\KOMAoptions{cleardoublepage=empty, parskip=half}
